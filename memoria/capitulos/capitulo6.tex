\chapter{Conclusiones}
\label{cap:capitulo6}

En este último capítulo se resumirán los objetivos conseguidos y, con ello, las conclusiones finales a las que se ha llegado en este trabajo. Además, se comentarán cuales han sido las competencias y conocimientos adquiridos durante el desarrollo, así como mencionar las líneas futuras.

\section{Consecución de objetivos}

Se destaca que se ha conseguido cumplir el objetivo principal de este trabajo: desarrollar una herramienta de reconocimiento de emociones que sea capaz de funcionar en un sistema robótico de bajo coste bajo el entorno ROS. Como se ha mostrado en la Sección \ref{sec:integración_en_ros}, el sistema de detección de emociones está funcionando bajo ROS Noetic a un rendimiento medio de 13.18 fps, esto es, en tiempo real.\\

Además, en el Capítulo \ref{cap:capitulo2} se dividió el objetivo general en una serie de subobjetivos y todos ellos se han cumplido:

\begin{itemize}
    \item \textit{Estudiar el estado del arte y elegir la técnica más óptima.} Como se ha comentado en el Capítulo \ref{cap:capitulo4}, se ha escogido la técnica comprendida por los siguientes tres pasos: detección de puntos faciales, extracción de información de esos puntos faciales, clasificación de dicha información. Y se ha cumplido el objetivo de ser una técnica liviana y precisa.
    
    \item \textit{Optimizar y adaptar la técnica escogida en nuestra plataforma.} Para ello, se ha realizado un estudio en búsqueda del algoritmo de detección de puntos faciales más rápido y preciso capaz de funcionar en nuestra plataforma (Sección \ref{sec:estudio_puntos_faciales}). Además, se ha investigado profundamente el estado del arte en cuanto a técnicas óptimas para obtener información confiable de dichos puntos faciales que, posteriormente, nos permitiera generar un buen dataset, saliendo victoriosa la construcción de una \textit{malla emocional} \cite{mediapipe_emotions} que proporciona información angular de las emociones (Sección \ref{sec:extraer_informacion}).
    
    \item \textit{Generar un dataset de valor.} Se buscó un dataset de imágenes que sirviera de base para posteriormente generar el nuestro con información angular obtenida de dichas imágenes (Sección \ref{sec:generacion_dataset}). El dataset escogido ha sido The Extended Cohn-Kanade Dataset (CK+) \cite{Kanade1}\cite{Kanade2}. Se realizaron diversos estudios que nos han ayudado a decidir qué información angular era la más óptima a la hora de generar la base de datos (Sección \ref{sec:estudio_cantidad_de_angulos}), y, posteriormente, más estudios que nos han indicado cual de los datasets generados nos ofrecía más precisión en el entrenamiento de nuestro modelo (Sección \ref{sec:estudio_dataset_optimo}).
    
    \item \textit{Realizar el entrenamiento del modelo.} Se han utilizado tres algoritmos de clasificación para entrenar nuestro modelo (KNN, SVM, MLP). A la hora de buscar los parámetros óptimos de cada uno de los algoritmos y también para testear la precisión en su entrenamiento, se ha utilizado la técnica de validación cruzada K-Fold Stratified. Finalmente, los tres modelos entrenados poseen un porcentaje de acierto del 95\% y todo el proceso de su entrenamiento se relata en la Sección \ref{sec:entrenamiento}.
    
    \item \textit{Integrar el sistema en ROS.} Tal como se muestra en la Sección \ref{sec:integración_en_ros}, se ha conseguido instalar ROS Noetic bajo Raspberry Pi OS y se ha creado un paquete llamado \verb|emotion_detection_ros| que porta la herramienta y la integra totalmente dentro del ecosistema ROS. En la Sección \ref{sec:uso_rendimiento_ros} se muestra el funcionamiento y rendimiento final del sistema, 13.18 fps de media.
\end{itemize}

\section{Competencias adquiridas}

Durante el desarrollo de este trabajo se han adquirido diferentes competencias y conocimientos, los cuales se listan a continuación:

\begin{itemize}
    \item Soltura a la hora de buscar y leer artículos de investigación.
    
    \item Profundización en el uso de un sistema de bajo coste Raspberry Pi y sus periféricos, así como su sistema operativo.
    
    \item Tomar decisiones tras realizar estudios que las validen.
    
    \item Experiencia usando librerías de tratamiento de datos y Machine Learning en Python (NumPy, pandas, Scikit-Learn).
    
    \item Aprendizaje de técnicas avanzadas de entrenamiento (validación cruzada KFold).
    
    \item Ampliación de conocimientos en ROS.
\end{itemize}

\section{Valoración final y líneas futuras}

Se ha desarrollado una herramienta que ofrece un servicio de detección de emociones a los desarrolladores de robótica de bajo coste, permitiéndoles avanzar en la investigación de la Interacción Humano-Robot. Es por ello que se plantean las siguientes líneas futuras que se podrían seguir a raíz de este trabajo:

\begin{itemize}
    \item Implementar aplicaciones de HRI que se ayuden de esta herramienta.
    
    \item Portar el sistema a otras plataformas para dar servicio en otros campos de investigación. Por ejemplo, incorporarlo en cines y así estudiar las reacciones del público a las distintas escenas de una película. Gracias a ser una herramienta que necesita pocos recursos, no es complicado incorporarla en otros sistemas.
    
    \item Sería interesante añadir nuevas funcionalidades a la herramienta, como por ejemplo, que sea capaz de detectar el género o predecir la edad del sujeto a evaluar.
\end{itemize}

