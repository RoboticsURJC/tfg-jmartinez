\cleardoublepage

\chapter*{Abstract\markboth{Abstract}{Abstract}}

Service robotics ---a broad branch of research within robotics--- is characterized by offering useful services to humans. Within this branch we find Human Robot Interaction or HRI, a field of study that arises from the need for robots to be able to collaborate and live with us, humans. For this purpose, Artificial Vision plays a fundamental role, which, together with Machine Learning, provides the robot with qualities such as facial and emotion recognition, giving it capabilities that allow it to interact appropriately according to the context.\\

Machine Vision tasks that make use of Machine Learning, usually demand a high computational load, and not all robot developers have enough money or even space within their automatons to get a large computational station. That is why in this work we have developed an emotion detection tool capable of running in a low-cost system, that is, in an embedded system characterized by its small size and low price.\\

Emotion detection has been carried out by training classification algorithms (KNN, SVM and MLP) with angular data extracted from facial expressions produced by human emotions (anger, happy, sadness and surprise). The angular data have been obtained from the construction of an \textit{emotional mesh}, from facial points, based on the Facial Coding System (FACS). The hardware used has been the Raspberry Pi 4 Model B board, and the final tool has been integrated in ROS Noetic to facilitate its use in robotic systems.\\

Several studies have been carried out in order to find the best optimization for the tool, as well as to perform final tests to verify its operation and performance in frames per second.