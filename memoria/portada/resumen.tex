\cleardoublepage

\chapter*{Resumen\markboth{Resumen}{Resumen}}

La robótica de servicio ---amplia rama de investigación dentro de la robótica--- se caracteriza por ofrecer servicios útiles para los humanos. Dentro de dicha rama encontramos la Interacción Humano Robot o HRI, campo de estudio que nace con la necesidad de que los robots sean capaces de colaborar y vivir con nosotros, los humanos. Para ello, la Visión Artificial juega un papel fundamental, que sumada al Machine Learning, proporcionan al robot cualidades como el reconocimiento facial y de emociones, dotándole de capacidades que le permiten interactuar de forma adecuada según el contexto.\\

Las tareas de Visión Artificial que hacen uso de Machine Learning, suelen demandar una alta carga computacional, y no todos los desarrolladores de robots poseen el suficiente dinero o incluso espacio dentro de sus autómatas como para hacerse con una gran estación de cómputo. Es por ello que en este trabajo se ha desarrollado una herramienta de detección de emociones capaz de funcionar en un sistema de bajo coste, o lo que es lo mismo, en un sistema empotrado o embebido caracterizado por su tamaño reducido y precio cometido.\\

La detección de emociones se ha llevado a cabo mediante el entrenamiento de algoritmos de clasificación (KNN, SVM y MLP) con datos angulares extraídos de las expresiones faciales producidas por emociones humanas (enfado, felicidad, tristeza y sorpresa). Los datos angulares se han obtenido de la construcción de una \textit{malla emocional}, a partir de puntos faciales, basada en el Sistema de Codificación Facial (FACS).
El hardware usado ha sido la placa Raspberry Pi 4 Model B y la herramienta final ha sido integrada en ROS Noetic para facilitar su uso en sistemas robóticos.\\

Se han realizado varios estudios con el afán de buscar la mayor optimización para la herramienta, además de realizar pruebas finales en las que se verificase su funcionamiento y el rendimiento en fotogramas por segundo.
